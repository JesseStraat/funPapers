\documentclass{article}
\usepackage[lang = english, numbering = section, nocount]{JesseTeX}
\usepackage{multicol}

\title{Categorical Hamiltonian mechanics}
\author{Jesse Straat\thanks{Utrecht University}}
\date{}

\newcommand\numberthis{\addtocounter{equation}{1}\tag{\theequation}}

\begin{document}
\maketitle

\begin{multicols}{2}
	\section{Hamiltonian mechanics}
	%% A short review on Hamiltonian mechanics to familiarise a mathematical audience
	In this first section, we will shortly review the basic concepts of Hamiltonian mechanics. Let us start by defining some Euclidean space \(\R^n\), which we will refer to as ``position space''. Additionally, we want to define a ``momentum space'', which is also equal to \(\R^n\).
	Together, these two spaces form phase space \(\R^{2n}\), whose coordinates are given by \(q_1,\,\dots,\,q_n,\,p_1,\,\dots,\,p_n\). In classical mechanics, we use this phase space to describe the motion of objects. This is achieved using a path \(\gamma:\ \R\to\R^{2n}\), such that \(\gamma(t)\) gives us both a position and a momentum (N.B.: the domain of these paths represent time).
	Not every path is a possible one. In fact, to ensure determinism, we should only allow there to exist one path to pass through any given point.\\
	How do we find the unique path? The answer lies in the Hamiltonian, a continuously differentiable function \(H:\ \R^{2n}\to\R\), which can be understood as assigning to each point in phase space the energy associated with it (the Hamiltonian is equal to the total energy more often than not).
	Hamiltonian mechanics supplies us with a powerful tool for finding the unique path. For any continuously differentiable function \(f:\ \R^{2n}\to\R\), we have that
	\begin{equation}
		\dv{(f\circ\gamma)}{t} = \left.\sum_{i=1}^n \pdv{f}{q_i}\pdv{H}{p_i} - \pdv{f}{p_i}\pdv{H}{q_i}\right|_{\gamma} =: \{f,\,H\}\circ\gamma.
	\end{equation}
	\(\{\cdot,\ \cdot\}\) is called the Poisson bracket, and is an important tool in Hamiltonian mechanics.
	Now, to get equations of motion for our path \(\gamma\), we can choose \(f\) to be our coordinates \(q_i\) and \(p_i\). This gives us a set of \(2n\) first-order differential equations
	\begin{equation}
		\dv{(q_i\circ\gamma)}{t} = \{q_i,\,H\}\circ\gamma = \pdv{H}{p_i}\circ\gamma,
	\end{equation}
	\begin{equation}
		\dv{(p_i\circ\gamma)}{t} = \{p_i,\,H\}\circ\gamma = -\pdv{H}{q_i}.
	\end{equation}
	These are known as Hamilton's equations.\\
	To give an example of how to use these equations, consider a particle in \(n\) dimensions moving around freely without a potential energy.
	The Hamiltonian in this case is given by \(H = \sum_{i=1}^n\frac{{p_i}^2}{2m}\), where \(m\) is te particle's mass. We then find Hamilton's equations to be
	\begin{equation}
		\dv{q_i\circ\gamma}{t} = \frac{p_i\circ\gamma}{m},
	\end{equation}
	\begin{equation}
		\dv{p_i\circ\gamma}{t} = 0.
	\end{equation}
	Therefore, if the particle starts with some momentum \(\vec{p}\), it must conserve that value over the entire path.
	Moreover, the position of the particle over time is given by \(\vec{x}(\gamma(t)) = \frac{\vec{p}}{m}t\). This gives us our path.
	%
	\section{Symplectic geometry}
	%% An introduction to using symplectic geometry to generalise the concept of Hamiltonian mechanics
	%
	\section{Categorical Hamiltonian theories}
	%% The categorical description of Hamiltonian mechanics using symplectic geometry
\end{multicols}
\end{document}