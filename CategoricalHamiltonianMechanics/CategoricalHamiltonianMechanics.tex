\documentclass{article}
\usepackage[lang = english, numbering = section, nocount]{JesseTeX}
\usepackage{multicol}

\title{Categorical Hamiltonian mechanics}
\author{J.R.H. Straat\thanks{Utrecht University}}
\date{}

\newcommand\numberthis{\addtocounter{equation}{1}\tag{\theequation}}
\newcommand{\category}[1]{{\normalfont\textbf{#1}}}

\begin{document}
\maketitle

\begin{multicols}{2}
	\section{Hamiltonian mechanics}
	%% A short review on Hamiltonian mechanics to familiarise a mathematical audience
	In this first section, we will shortly review the basic concepts of Hamiltonian mechanics. Let us start by defining some Euclidean space \(\R^n\), which we will refer to as ``position space''. Additionally, we want to define a ``momentum space'', which is also equal to \(\R^n\).
	Together, these two spaces form phase space \(\R^{2n}\), whose coordinates are given by \(q_1,\,\dots,\,q_n,\,p_1,\,\dots,\,p_n\). In classical mechanics, we use this phase space to describe the motion of objects. This is achieved using a path \(\gamma:\ \R\to\R^{2n}\), such that \(\gamma(t)\) gives us both a position and a momentum (N.B.: the domain of these paths represent time).
	Not every path is a possible one. In fact, to ensure determinism, we should only allow there to exist one path to pass through any given point.\\
	How do we find the unique path? The answer lies in the Hamiltonian, a continuously differentiable function \(H:\ \R^{2n}\to\R\), which can be understood as assigning to each point in phase space the energy associated with it (the Hamiltonian is equal to the total energy more often than not).
	Hamiltonian mechanics supplies us with a powerful tool for finding the unique path. For any continuously differentiable function \(f:\ \R^{2n}\to\R\), we have that
	\begin{equation}
		\dv{(f\circ\gamma)}{t} = \left.\sum_{i=1}^n \pdv{f}{q_i}\pdv{H}{p_i} - \pdv{f}{p_i}\pdv{H}{q_i}\right|_{\gamma} =: \{f,\,H\}\circ\gamma.
	\end{equation}
	\(\{\cdot,\ \cdot\}\) is called the Poisson bracket, and is an important tool in Hamiltonian mechanics.
	Now, to get equations of motion for our path \(\gamma\), we can choose \(f\) to be our coordinates \(q_i\) and \(p_i\). This gives us a set of \(2n\) first-order differential equations
	\begin{equation}
		\dv{(q_i\circ\gamma)}{t} = \{q_i,\,H\}\circ\gamma = \pdv{H}{p_i}\circ\gamma,
	\end{equation}
	\begin{equation}
		\dv{(p_i\circ\gamma)}{t} = \{p_i,\,H\}\circ\gamma = -\pdv{H}{q_i}.
	\end{equation}
	These are known as Hamilton's equations.\\
	To give an example of how to use these equations, consider a particle in \(n\) dimensions moving around freely without a potential energy.
	The Hamiltonian in this case is given by \(H = \sum_{i=1}^n\frac{{p_i}^2}{2m}\), where \(m\) is te particle's mass. We then find Hamilton's equations to be
	\begin{equation}
		\dv{q_i\circ\gamma}{t} = \frac{p_i\circ\gamma}{m},
	\end{equation}
	\begin{equation}
		\dv{p_i\circ\gamma}{t} = 0.
	\end{equation}
	Therefore, if the particle starts with some momentum \(\vec{p}\), it must conserve that value over the entire path.
	Moreover, the position of the particle over time is given by \(\vec{x}(\gamma(t)) = \frac{\vec{p}}{m}t\). This gives us our path.
	%
	\section{Symplectic geometry}
	%% An introduction to using symplectic geometry to generalise the concept of Hamiltonian mechanics
	We would now like to generalise the concepts of Hamiltonian mechanics to other manifolds, rather than just Euclidean space. One way of doing this is through symplectic manifolds.
	\begin{definition}
		A symplectic manifold \((M,\,\omega)\) is a pair consisting of a differentiable manifold \(M\) and a non-degenerate \(2\)-form \(\omega\) on \(M\). We say that \(\omega\) is the symplectic form of the symplectic manifold.
	\end{definition}
	Let us dissect this definition. The symplectic form \(\omega\) is a \(2\)-form on \(M\). This means that it is an asymmetric bilinear map that takes two vector fields \(V,\,W\in\mathfrak{X}(M)\) on \(M\), and maps them to some differentiable real function on \(M\). Moreover, since it is non-degenerate, we know that if \(\omega(V,W) = 0\) for all \(V\), then \(W=0\).
	This sounds a lot like the the Poisson bracket, which was an asymmetric bilinear map that takes two real functions \(f,\,g\) on \(\R^{2n}\), and maps them to another real function on \(\R^{2n}\). Moreover, if \(\{f,\,g\}=0\) for all \(f\), then \(g\) had to be constant.
	In fact, from the symplectic form, we can derive an object with the same properties as the Poisson bracket
	% TODO: What other properties does the Poisson bracket have that should be met?
	\begin{equation}
		\{f,\,g\} = \omega(X(f),\,X(g)).
	\end{equation}
	Here, \(X\) is some linear functor that maps real functions to vector fields over \(M\), such that constant functions are mapped to the zero-vector field. We call \(X\) a Hamiltonian vector field.\\
	From this definition, if we define some function \(H:\ M\to\R\), the Hamiltonian, we can, as before, find the equations of motion for the true paths \(\gamma\) over \(M\),
	\begin{equation}
		\dv{(f\circ\gamma)}{t} = \{f,\,H\}\circ\gamma
	\end{equation}
	for any differentiable real function \(f\) on \(M\).\\
	The take-away here is that it is possible to formulate a way of deriving equations of motion (a ``Hamiltonian theory'', if you will), given only a Hamiltonian vector field.\\
	Note that, being a \(2\)-form, we can reframe \(\omega\) as being a non-degenerate asymmetric function \(\mathfrak{X}(M)\to \Omega^1(M)\), which then allows us to define an invere function \(\Omega:\ \Omega^1(M)\to\mathfrak{X}(M)\).
	This gives us a Hamiltonian vector field:
	\begin{equation}
		X(f) := \Omega(\dd f).
	\end{equation}
	In fact, this specific Hamiltonian vector field is the one that is used in regular Hamiltonian mechanics.
	%
	\section{Categorical Hamiltonian theories}
	%% The categorical description of Hamiltonian mechanics using symplectic geometry
	In this final section, we will describe the processes from the previous sections in an abstract, categorical manner. As you would expect, this first requires us to define a fitting category.
	\begin{definition}
		Let \(M\) be a differentiable manifold. \(\category{Ham}_M\) is the category of Hamiltonian systems on \(M\), i.e.,
		\begin{itemize}
			\item its objects are pairs \((\omega,\,H)\), where \((M,\,\omega)\) is a symplectic manifold, and \(H\) is some differentiable real function on \(M\);
			\item the morphisms \((\omega,\,H) \to (\eta,\,K)\) are differentiable functions \(f:\ M\to M\) such that \(f^*\eta = \omega\) and \(f^*K = H\).
		\end{itemize}
	\end{definition}
	As usual with morphisms, these morphisms were defined exactly to preserve the defining structures of the objects, i.e., the symplectic forms and the Hamiltonians.\\
	Now, we want to derive equations of motion from these Hamiltonian systems. To this end, we will define another category.
	\begin{definition}
		\(\category{EOM}_M\) is the category of equations of motion, defined such that
		\begin{itemize}
			\item its objects are functions \(\partial\) that maps differentiable functions on \(M\) to other differentiable functions on \(M\);
			\item its morphisms \(\partial \to \partial'\) are differentiable functions \(F:\ M\to M\) such that \(f^*\partial'(f) = \partial(F^*f)\) for all differentiable functions \(f\) on \(M\).
		\end{itemize}
	\end{definition}
	Here, we deliberately chose the name \(\partial\) to reflect the fact that these are the functions that give us the equations of motion:
	\begin{equation}
		\dv{f\circ\gamma}{t} = \partial(f)\circ\gamma.
	\end{equation}
	Essentially, the objects of \(\category{EOM}_M\) consists of all functions that defines a set of well-defined equations of motion.\\
	Now, with these tools in hand, we can define a process that turns Hamiltonian system into a manifold and its equations of motion.
	\begin{definition}
		A Hamiltonian theory on \(M\) is a functor from \(\category{Ham}_M\) to \(\category{EOM}_M\).
	\end{definition}
\end{multicols}
\end{document}