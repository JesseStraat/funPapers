%% This paper is not necessarily for fun. It was for a course on open science, and was written as part of an assignment on science communication.
%% However, I have chosen to include it in the funpaper repository to make sure the paper remains open source, in spirit of the course.
%% Note that this article is written in the spirit of a PopSci news article, *not* a scientific paper for publishing.
%% The article is based on arXiv:2308.06669 [quant-ph]

\documentclass[12pt]{article}
\PassOptionsToPackage{backend=biber,sorting=none,style=nature}{biblatex}
\usepackage[lang = english, nocount, customcite]{JesseTeX}
\importjesselibrary{tikz}
\usepackage{multicol}
\usepackage{float}
\usepackage{docmute}
\usepackage{pgfplots}
\pgfplotsset{compat=1.18}

\addbibresource{references.bib}

\title{Our Mathematical Model of Quantum Mechanics is Unphysical}      % Preliminary title
\author{J.R.H. Straat}
\date{}

\newcommand{\category}[1]{{\normalfont\textbf{#1}}}

\begin{document}
\renewcommand{\abstractname}{}
\maketitle
%
\begin{abstract}
	The mathematical concept used to model quantum mechanics is too broad. It predicts the existence of quantum states that are not physical.
\end{abstract}
%
\begin{multicols}{2}
	%% Introductory part
    In the early days of quantum mechanics, there were two major models: matrix mechanics and wave mechanics. Eventually, the two models were proven to be mathematically equivalent, but neither provided a mathematically rigorous framework.\\
    This all changed in 1932, when mathematician and physicist John von Neumann published his ground-breaking \textit{Mathematische Grundlagen der Quantenmechanik}\cite{von_neumann_mathematische_1996}. Only three years after defining the \textit{Hilbert space}, named after his academic advisor David Hilbert, von Neumann used it to provide a mathematical foundation for quantum mechanics.\\
    However, a recent paper\cite{carcassi_unphysicality_2023} points out that the Hilbert space is a flawed model for quantum mechanics, since some of its predictions are unphysical.
    %% Mathematics and physics
    \section*{Mathematical models}
    First, we should talk about the role of mathematics in physics. In an attempt to describe reality, a physicist may create a theory: a collection of physical laws. The theory may be ambiguous and attempts to apply the theory may be founded on poorly-defined processes. For example, despite its experimental success, the path integral approach to quantum field theory is known to have no formal mathematical definition.\\
    This is where mathematical physicists come in. By creating a mathematical model, we formulate the rules of the physical theory in an unambiguous mathematical language. This allows us to remove ambiguities, apply existing knowledge of mathematics and discover paradoxes, encouraging new developments.
    %% What does it mean to be (un-)physical?
    \section*{(Un)physicality}
    Let us now talk about the word ``unphysical''. What does it mean for a mathematical model to be unphysical? Recall that we started with a pure physical theory, and connected mathematics to it to ensure clearness and encourage development. However, a poor choice of mathematical theory may not fit the physical theory.\\
    %Most of the time, mathematical structures precede the physical theory for which they are developed. For example, Riemann is attributed with creating the concept of a manifold in 1854, seventy years before Einstein's theory of general relativity (although the precise definition did not exist until the early 30's).\\    % This paragraph seems irrelevant
    A mathematical model is \textit{physical} if its predictions match the physical theory exactly. An \textit{unphysical} model may either lack parts of the physical theory, or predict too many things. By adding the mathematics to our theory, we lose the physics in the process.
    %% Vague description of Hilbert spaces
    \section*{Hilbert spaces}
    Let us apply our knowledge to Hilbert spaces, von Neumann's mathematical model of quantum mechanics. The first of the properties of a Hilbert space is that it is a vector space, which roughly means that the sum of any two quantum states must also be a quantum state. Physically, this is precisely the concept of superposition! Hence, the property of being a vector space is physical. The second property, that of the inner product, is also shown to be physical by the authors of the paper.\\
    Now, the Hilbert space's troublesome and final property is completeness. This means that given certain infinite sequences of quantum states (called Cauchy sequences), there must exist a state that is the limit of the sequence.\\
    The authors of the article show that completeness gives quantum states that are physically impossible. For example, we can create a state whose position expectation value diverges: it is not defined. It may even go to infinity (see, for example, figure \ref{fig:sketch}). For physical reasons, these states should of course be excluded from our theory. The completeness condition, and hence the Hilbert space, is therefore unphysical.
    %% Alternatives
    \section*{Alternatives}
    The question remains whether there exists a physical alternative to Hilbert spaces. One popular alternative to Hilbert spaces is the rigged Hilbert space\cite{de_la_madrid_role_2005}. However, as the authors mention, rigged Hilbert spaces are
    \begin{figure}[H]
        \centering
        \documentclass{standalone}
\usepackage{pgfplots}
\pgfplotsset{compat=1.18}
\usepackage{tikz}
\usetikzlibrary{calc}

\begin{document}
    \def\sketchmax{3}
    \begin{tikzpicture}
        \begin{axis}[xmin=0,xmax=2^\sketchmax,yticklabel style={/pgf/number format/fixed},xtick pos=left,ytick pos=left,xlabel={$x$},ylabel={$\psi(x)$}]
            \addplot[domain=0:0.5] {0};
            \foreach \muVal in {0,1,2,...,\sketchmax}{
                \edef\temp{\noexpand\addplot[domain=2^\muVal-0.5:2^\muVal+0.5,samples=100] {1/sqrt(0.0665431)*1/sqrt(2)^(\muVal)*e^(-1/(1-4*(2^\muVal-x)^2))};
                \noexpand\addplot[domain=2^\muVal+0.5:2^(\muVal+1)-0.5,samples=100] {0};}
                \temp
            }
        \end{axis}
    \end{tikzpicture}
\end{document}
        \caption{A one-dimensional example of a quantum wave whose position expecation value diverges to infinity.}
        \label{fig:sketch}
    \end{figure}
    also Hilbert spaces, so the same issues remain.\\
    The authors do leave us with another possible alternative called Schwartz spaces\cite{becnel_schwartz_2015}, which are more restrictive than the Hilbert space. Indeed, they are formed exactly by removing all states whose (combinations of) momentum and position expectations diverge, thus losing completeness. However, no conclusion is drawn on the physicality of Schwartz spaces, and further research should be conducted to investigate the possibility.
    %% Word count: 625
\end{multicols}
\printbibliography
\end{document}