%% This paper is not necessarily for fun. It was for a course on open science, and was written as part of an assignment on science communication.
%% However, I have chosen to include it in the funpaper repository to make sure the paper remains open source, in spirit of the course.
%% Note that this article is written in the spirit of a PopSci news article, *not* a scientific paper for publishing.
%% The article is based on arXiv:2308.06669 [quant-ph]

\documentclass[12pt]{article}
\usepackage[lang = english, numbering = section, nocount]{JesseTeX}
\usepackage{multicol}

\addbibresource{references.bib}

\title{Our Mathematical Model of Quantum Mechanics is Unphysical}      % Preliminary title
\author{J.R.H. Straat}
\date{}

\newcommand{\category}[1]{{\normalfont\textbf{#1}}}

\begin{document}
\maketitle
%
\begin{abstract}
	The mathematical concept used to model quantum mechanics is too broad. It predicts the existence of quantum states that are not physical.
\end{abstract}
%
\begin{multicols}{2}
	%% Introductory part
    In the early days of quantum mechanics, there were two major models of quantum mechanics: matrix mechanics and wave mechanics. Eventually, the two models were proven to be mathematically equivalent, but neither provided a mathematically rigorous framework.\\
    This all changed in 1932, when mathematician and physicist John von Neumann published his ground-breaking \textit{Mathematische Grundlagen der Quantenmechanik}\cite{von_neumann_mathematische_1996}. Only three years after defining the ``Hilbert space'', named after his academic advisor David Hilbert, von Neumann used it to provide a mathematical foundation for quantum mechanics.\\
    However, a recent paper\cite{carcassi_unphysicality_2023} points out that the Hilbert space is a flawed model for quantum mechanics, since some of its predictions are ``unphysical''.
    %% Why mathematical descriptions matter
    \section{Why physics needs maths}
    %% Vague description of Hilbert spaces
    \section{Hilbert spaces}
    %% What does it mean to be (un-)physical?
    \section{(Un-)physicality}
    %% Conclusion
    \section{Conclusion}
\end{multicols}
\printbibliography
\end{document}